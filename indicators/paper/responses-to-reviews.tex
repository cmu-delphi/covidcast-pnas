\documentclass[11pt]{article}

\usepackage[T1]{fontenc}
\usepackage{geometry}
\usepackage{microtype}

\usepackage{newtxmath}
\usepackage{newtxtext}

\title{Response to reviewers}
\author{}
\date{\today}

\begin{document}
\maketitle

\section*{Editor}

\begin{quote}
  The reviewers both saw high value in the database and regarded the paper as
  important and relevant. Reviewer 2 had several high-level comments about the
  current framing of the results and offered a number of suggestions for the
  authors to consider additional or modified analyses that would strengthen the
  paper. I would like to invite the authors to submit a revised version of the
  manuscript including a detailed point-by-point response to the reviewer
  comments, particularly where Reviewer 2 has suggested need for either further
  analyses or pointed to notable limitations. While the authors should use their
  discretion regarding whether additional analyses would be within scope for the
  revision, it will be important to respond to the suggestions in either case
  and at least to expand on limitations as relevant in the revised manuscript.
\end{quote}
TODO

\section*{Reviewer 1}

\begin{quote}
  In section 1.B, you mention non-trivial signal processing and direct the
  reader to github to learn more. I would like to see more discussion of how
  each source was processed. If this is not possible due to length constraints,
  where can the reader find more details? Is there documentation at the Delphi
  website that explains how the data from each source is processed? In a quick
  search at the Delphi website, I was not able to find information about how the
  survey data (for example) were weighted. These details will be important to
  those using the data.
\end{quote}
Thanks, this is an important point. We have added a sentence to Section 1.B to
make clear that detailed documentation, including such details as the survey
weighting procedures and the day-of-week adjustments used in other data sources,
is available on the public documentation site.

\section*{Reviewer 2}

\begin{quote}
  This paper describes a diverse public database of signals related to COVID-19
  prevalence, including symptom survey data, mobility, at-home tests, and
  hospitalizations. The database is unique, important, and useful, especially in
  the current surveillance-limited context of the United States. However, the
  main analyses presented in the results section did not feel cohesive, and the
  ``story'' of the paper felt incomplete. While I respect the challenge of
  dividing work among multiple related papers, I think that a key goal of
  revisions might be to add further independent analyses of the data described
  that emphasize its utility beyond comparisons to cases or add additional
  analyses to build a unique cohesive story for this paper.
\end{quote}
We appreciate this point and agree the message may have gotten split among
manuscripts unnecessarily. We have made some changes to the Results section
accordingly. Most notably, we have replaced Figure 6 with one that more clearly
shows a potential use case of the data presented beyond forecasting of cases; it
now shows now data sources in the API could have aided in planning for the COVID
vaccination campaign.

TODO any additional changes?

\begin{quote}
  The main value of the data discussed in the paper seems to focus on its
  potential utility in forecasting, which guides the main comparisons to
  observed case burden and primary outcome of ``signal value'' in results and in
  the appendix. However, there are a number of important limitations to the
  results presented that impede this message.

  Most concrete results on this point seem to be presented in another paper.
  Correlations in the paper are focused on current case burden (signal vs. cases
  both at time t), rather than prediction for forecasts.
\end{quote}
TODO

\begin{quote}
  Even from a nowcasting perspective, comparisons are all benchmarked against
  cases which are observable in real-time.
\end{quote}
TODO

\begin{quote}
  The strength of observed correlations change (non-monotonically) over time,
  and the implications of this are not discussed.
\end{quote}
While it is difficult to fully explain the changes in correlation---they are
likely due to a variety of changes in reporting practices, health-seeking
behavior, survey response patterns, and so on---one key point is that the
strength of the correlation depends on the overall differences between counties.
That is, if counties have very similar COVID case rates, it is difficult to
achieve a high Spearman correlation between case rates and signals, because
getting the rank correct is difficult. If counties have widely varying COVID
case rates, a signal that provides information about cases will be able to
achieve a higher Spearman correlation.

We have slightly expanded our discussion of Figure 2 to make this point, though
we have tried to keep the discussion brief so as not to sidetrack the main
discussion of the manuscript.

\begin{quote}
  Signal strength is compared only to cases (rather than downstream outcomes
  like hospitalization or death).
\end{quote}
We have extended the Supplementary Information to include more detailed
comparisons between our signals and reported COVID hospitalizations, including
comparisons of correlations. These show broadly similar results as the COVID
case analyses. Section 2.B now points to the Supplement for interested readers
to examine these analyses.

\begin{quote}
  The case studies of signal utility that are provided seem stylized.
  Policymakers would *know* in real-time when backlogged cases are reported and
  can adjust case numbers accordingly or when cases are not being reported due
  to holiday backlogs.
\end{quote}
Unfortunately this is not the case: Backlogs and reporting delays are generally
poorly understood and can happen unexpectedly, and users of these datasets often
have to make informed guesses about changes. For instance, many backlogs are due
to retrospective audits of records that uncover cases and deaths that were
incorrectly classified, and can affect case and death estimates months in the
past. Until the audits are complete, policymakers would not know about the
corrections, their magnitude, or the affects they'd have on policy decisions.

Even for predictable backlogs, such as those that occur during major holidays,
the backlog still means policymakers do not have an important signal: If case
and death reporting is suspended for the holiday, knowing that the suspension is
due to the holiday does not help a policymaker judge whether cases are
increasing or decreasing and whether urgent action is required. Outside data
sources would be needed to inform that decision.

We have adjusted the language in Section 2.C to make these points clearer.

\begin{quote}
  As a result, it might be useful to focus quantitative analyses in this paper
  on other attributes of the data or expand analyses to include, e.g.:
  \begin{itemize}
  \item descriptive statistics from each data source, expressed in the units of
    that data source
  \item analysis of how signals differ
  \item analysis of case correlations over time due to discussion of how they
    have evolved over time -- e.g. due to changes in vaccination, testing, etc.
  \item for case studies: I think it would be more powerful to either highlight
    an instance issues were not apparent in the absence of other signals or to
    provide more in depth analysis about a specific policy conclusion that would
    result from use of these signals (or specific examples of
    geographic/temporal variation that the signal data can uniquely inform).
  \end{itemize}
  A compelling framing for the results section might be to include general
  descriptive statistics and then pick case studies that illustrate 2-3 uses of
  the data.
\end{quote}
We appreciate the interest in additional analyses, and our replacement of Figure
6 is partly intended to address this concern about choice of case studies. It
represents an example showing how these data sources can reveal points not
available in other standard data sources, such as case and death data, and serve
to aid public health decision-making.

We have chosen not to include additional descriptive statistics or comparisons
between signals for two reasons. First, given the large number of signals
available in our API, a table of descriptive statistics would be prohibitively
large; this is particularly difficult if one accounts for time and geography by
presenting statistics over time or across space. Second, it's not clear to us
what a reader would gain from such statistics. Because the data is freely
available, a reader interested in any specific signal can quickly investigate it
and explore whatever aspects of it are of interest.

\begin{quote}
  I think that descriptions in results of ``what the data could be used for''
  are less compelling than more specific examples would be. For instance, the
  section on revised estimates over time seems to echo what was in the methods
  and implications don't really hit home without connection to changes in
  specific forecasts.
\end{quote}
TODO

\begin{quote}
  Some of the verb tenses (e.g. line 33) feel awkward: ``The Delphi Group
  **works** (worked) with partner organizations and34 public data sets to build
  a massive database of indicators35 tracking COVID-19 activity and other
  relevant phenomena36 in the United States, which has been publicly available
  and37 continuously updated since April 2020.''
\end{quote}
We have tweaked this sentence and a few others, though we note that the present
tense is accurate in many cases, because data collection and aggregation is
ongoing.

\begin{quote}
  It would be helpful to end the introduction with a description of the
  contribution of this paper, rather than other papers.
\end{quote}
We have adjusted the introduction to conclude with the contribution of this
paper, to set it in context against the other papers.

\begin{quote}
  I think it would be useful to replace ``massive'' with a more precise
  adjective unless it is standard in the literature.
\end{quote}

TODO

\begin{quote}
  Similarly, the verb ``ingest'' also feels odd to refer to data sources.
\end{quote}
``Ingest'' is commonly used in the data science community to refer to the
process of extracting data and loading it into a central database; but we agree
this use of the term may not be widespread among our audience, so we have
adjusted the language to avoid using field-specific jargon.

\begin{quote}
  I'm not sure that it is common practice to repeatedly refer to a sponsoring
  research group throughout an academic article.
\end{quote}
We have adjusted the language used throughout, so references to the research
group are limited to where they are most useful to the reader.

\begin{quote}
  Descriptions of statistical analyses (e.g. correlation methods) feel more
  appropriate for the methods section, not the results section.
\end{quote}
We are open to changes if the Editor feels they're necessary to match typical
\textit{PNAS} style, but we feel that in our paper, it makes the most sense for
the Methods to describe the methods underlying the data we gather and aggregate,
since that data is the main focus of the article. The Results focus on
illustrations of the data's utility, including their correlations with other
COVID indicators.

\end{document}
