\documentclass[9pt,twoside,lineno]{pnas-new}
% Use the lineno option to display guide line numbers if required.

\templatetype{pnassupportinginfo}

% Math
\def\P{\mathbb{P}}
\def\cor{\mathrm{cor}}
\def\Quantile{\mathrm{Quantile}}
\def\logit{\mathrm{logit}}
\def\dist{\mathrm{dist}}
\def\WIS{\mathrm{WIS}}
\def\AUC{\mathrm{AUC}}
\def\CCF{\mathrm{CCF}}
\newcommand{\indicator}[1]{\mathbf{1}\left(#1\right)}

% Figures and tables
\usepackage{xurl}
\usepackage{microtype}
\usepackage{booktabs}
\usepackage{caption}
\usepackage{subcaption}
\usepackage{xcolor}
\newcommand{\attn}[1]{\textcolor{red}{[ATTN: #1]}}

\makeatletter 
\renewcommand\@biblabel[1]{#1} 
\makeatother

% indicators
\newcommand{\chngcli}{CHNG-CLI}
\newcommand{\chngcov}{CHNG-COVID}
\newcommand{\dv}{DV-CLI}
\newcommand{\ar}{AR}
\newcommand{\fb}{CTIS-CLI-in-community}
\newcommand{\gs}{Google-AA}


\providecommand{\tightlist}{%
  \setlength{\itemsep}{0pt}\setlength{\parskip}{0pt}}


\title{Your main manuscript title}
\author{Author1, Author2 and Author3 (complete author list)}
\correspondingauthor{Corresponding Author name.\\E-mail: author.two@email.com}

\begin{document}



\instructionspage

\maketitle

\SItext

All our text goes here. Reference figures with the R chuck label as
Figure\textasciitilde{}\ref{fig:fcast-finalized}.

All figures go on their own page after all of the text\ldots{}

Edit \texttt{pnas-suppl-template.tex} for the correct Author list and
Title.

\hypertarget{examining-the-relative-advantage-of-using-finalized-rather-than-vintage-data}{%
\section{Examining the relative advantage of using finalized rather than
vintage
data}\label{examining-the-relative-advantage-of-using-finalized-rather-than-vintage-data}}


The goal of this section is to quantify the effect of
not properly accounting for the question of ``what was known when'' in
performing retrospective evaluations of forecasters.   Figures \ref{fig:fcast-finalized} and \ref{fig:hot-finalized} show
what Figures 3 and 4 in the main paper would have looked like if we
had simply trained all models using the finalized data rather than
using vintage data.  This comparison can be seen most
straightforwardly in Figures \ref{fig:fcast-honest-v-finalized} and \ref{fig:hot-honest-v-finalized}, which show the ratio in performance between the vintage and finalized
versions.  When methods are
given the finalized version of the data rather than the version available at the time that the forecast would have been made, all methods
appear (misleadingly) to have better performance than they would
actually have had if run prospectively.  For example, for forecasting case rates
7-days ahead, the WIS of all methods is at least 8\% larger than what would have been
recorded using the finalized values of the data.  This effect
diminishes as the forecasting horizon increases, reflecting the fact
that these forecasters rely less heavily on recent data than very
short-term forecasters.  Crucially, some methods are
``helped'' more than others by the less scrupulous retrospective
evaluation, underscoring the difficulty of avoiding misleading
conclusions when performing retrospective evaluations of forecasters.

The \chngcli~indicator (along with the other claims-based signals) is the most
affected by this distinction, reflecting the latency in claims-based
reporting.  This supports the importance of efforts to provide ``nowcasts'' for claims signals
(which corresponds to a 0-ahead ``forecast'' of what the claims
signal's value will be once all data has been collected). Looking at
the \chngcli~and \dv~curves in Figure \ref{fig:fcast-finalized}, we
can see that they perform very similarly when trained on the finalized data.  This is reassuring
because they are in principle measuring the same thing (namely, the percentage of
  outpatient visits that are primarily about COVID-related symptoms).
  The substantial difference in their curves in Figure 3 of the
  main paper must therefore reflect their having very different
  backfill profiles.  
  
While \dv~is one of the least affected methods by using finalized
rather than vintage values, it is one of the most affected methods for the hotspot problem.
This is a reminder that the forecasting and hotspot problems are
distinct problems.  For example, the hotspot problem does not measure
the ability to distinguish between flat and downward trends.

Even the \ar~model is affected by this distinction, reflecting the
fact that the case rates themselves (i.e., the response values) are also
subject to revision.  The forecasters based on indicators are thus
affected both by revisions to the indicators and by revisions to the
case rates.  In the case of the \gs~model, in which we only used finalized
values for the \gs~indicator, the difference in performance can be
wholly attributed to revisions of case rates.


\hypertarget{aggregating-with-geometric-mean}{%
\section{Aggregating with geometric
mean}\label{aggregating-with-geometric-mean}}

In this section, we consider using the geometric mean instead of the
arithmetic mean when aggregating the weighted interval score (WIS) across location-time pairs.
There are three reasons why the geometric mean may be desirable.
\begin{enumerate}
\item  WIS is right-skewed, being bounded below by zero and having
  occasional very large values.   Figure\textasciitilde{}\ref{fig:wis-densities} illustrates that the
  densities appear roughly log-Gaussian.  The geometric mean is a
  natural choice in such a context since the relative ordering of forecasters is
  determined by the arithmetic mean of the {\em logarithm} of their WIS
  values.
\item In the main paper, we report the ratio of the mean WIS of a
  forecaster to the mean WIS of the baseline forecaster. Another
  choice could be to take the mean of the ratio of WIS values for the
  two methods. (This latter choice would penalize a method less for
  doing poorly where the baseline forecaster also does poorly.)
Using instead the geometric mean makes the order of aggregation and
scaling immaterial since the ratio of geometric means is the same as
the geometric mean of ratios.
\item If one imagines that a forecaster's WIS is composed of
  multiplicative space-time effects $S_{\ell,t}$ shared across all forecasters,
  i.e. $\WIS(F_{\ell,t,f},Y_{\ell,t})=S_{\ell,t}E_{f}$, then
  taking the ratio of two forecasters' geometric mean WIS values will
  effectively cancel these space-time effects.
\end{enumerate}

Figure\textasciitilde{}\ref{fig:fcast-adjusted} uses the geometric
mean for aggregation.  Comparing this with Figure 3 of the main paper,
we see that the main conclusions are largely unchanged; however,
\chngcli~now does appear better than \ar.  This behavior would be
expected if \chngcli's poor performance is attributable to a
relatively small number of large errors (as opposed to a large number
of moderate errors).  Indeed, Figure 5 of the main paper further
corroborates this, in which we see the heaviest left tails occuring
for \chngcli.

\hypertarget{bootstrap-results}{%
\section{Evaluating the implicit regularization hypothesis}\label{bootstrap-results}}

As explained in Section 2.B. of the main paper, a (somewhat cynical)
hypothesis for why we see benefits in forecasting and hotspot
prediction is that the indicators are not actually providing useful
information but they are instead simply acting as a sort of ``implicit
regularization,''  leading to shrinkage on the autoregressive
coefficients, leading to less volatile predictions.  To investigate
this hypothesis, we consider fitting  ``noise features'' that in truth
should have zero coefficients.  Recall (from the main paper) that at each forecast date, we
train a model on 6,426 location-time pairs.  Indicator models are
based on six features, corresponding to the three autoregressive terms
and the three lagged indicator values.  To form ``noise features''
we resample from other rows.  In particular, at each location $\ell$ and
time $t$, we replace the triplet $(X_{\ell,t}, X_{\ell,t-7},
X_{\ell,t-14})$ by the triplet $(X_{\ell^*,t^*}, X_{\ell^*,t^*-7},
X_{\ell^*,t^*-14})$, where $(\ell^*,t^*)$ is a location-time pair
sampled with replacement from the 6,426 location-time pairs.  Figures
\ref{fig:fcast-booted}--\ref{fig:hot-booted} show the results.  No
method exhibits a noticeable performance gain over the \ar~method,
leading us to dismiss the implicit regularization hypothesis.

\attn{Would be good for someone to double check that the bootstrap
  procedure is correctly described.}

\hypertarget{correlations-with-lagged-actuals}{%
\section{Correlations with lagged
actuals}\label{correlations-with-lagged-actuals}}

Alden's histograms are in Figure\textasciitilde{}\ref{fig:cor-wis-ratio}
and Figure\textasciitilde{}\ref{fig:cor-wis-ratio-m1}.

\hypertarget{upswings-and-downswings}{%
\section{Upswings and Downswings}\label{upswings-and-downswings}}

Logged version of Figure 5 in the manuscript is in
Figure\textasciitilde{}\ref{fig:upswing-histogram-logged}.

See also Table\textasciitilde{}\ref{tab:upswing-corr-table}.

\hypertarget{leadingness-and-laggingness}{%
\section{Leadingness and
laggingness}\label{leadingness-and-laggingness}}

Currently, both figures are in the manuscript. Probably just need text
here.

\hypertarget{examining-data-in-2021}{%
\section{Examining data in 2021}\label{examining-data-in-2021}}

In this section, we investigate the sensitivity of the results to the
period over which we train and evaluate the models.  In the main
paper, we end all evaluation on December 31, 2020.  Figures
\ref{fig:fcast-alldates} -- \ref{fig:hot-alldates} show how the
results would differ if we extended this analysis through \attn{XX,
  XX}, 2021. Comparing Figure \ref{fig:fcast-alldates} to Figure 3 of
the main paper, one sees that as ahead increases most methods now improve
relative to the baseline forecaster. When compared to other methods, \chngcli~appears much better than
it had previously; however, all forecasters other than \chngcov~and
\dv~are performing less well relative to the baseline than before.
These changes are likely due to the differing nature of the pandemic
in 2021, with flat and downward trends much more common than upward
trajectories.  Indeed, the nature of the hotspot prediction problem is
quite different in this period.  With a 21-day training window, it is
common for there to be very few hotspots in training.


\hypertarget{deprecated}{%
\section{Deprecated}\label{deprecated}}

There are a few blocks at the bottom (figures with Google symptoms only
and the old trajectory plots) that we can remove once we decide.

\begin{figure}

{\centering \includegraphics[width=\textwidth]{fig/fcast-finalized-1} 

}

\caption{Forecasting performance using finalized data. Compare to Figure 3 in the manuscript.}\label{fig:fcast-finalized}
\end{figure}

\clearpage

\begin{figure}

{\centering \includegraphics[width=\textwidth]{fig/hot-finalized-1} 

}

\caption{Hotspot prediction performance using finalized data. Compare to Figure 4 in the manuscript.}\label{fig:hot-finalized}
\end{figure}

\clearpage

\begin{figure}

{\centering \includegraphics[width=\textwidth]{fig/fcast-honest-v-finalized-1} 

}

\caption{Relative forecast WIS with vintage compared to finalized data. Using finalized data leads to overly optimistic performance.}\label{fig:fcast-honest-v-finalized}
\end{figure}

\clearpage

\begin{figure}

{\centering \includegraphics[width=\textwidth]{fig/hot-honest-v-finalized-1} 

}

\caption{Relative AUC with vintage compared to finalized data. Using finalized data leads to overly optimistic hotspot performance.}\label{fig:hot-honest-v-finalized}
\end{figure}

\clearpage

\begin{figure}

{\centering \includegraphics[width=\textwidth]{fig/wis-densities-1} 

}

\caption{Weighted interval score appears to more closely resemble a log-Gaussian distribution.}\label{fig:wis-densities}
\end{figure}

\clearpage

\begin{figure}

{\centering \includegraphics[width=\textwidth]{fig/fcast-adjusted-1} 

}

\caption{Relative forecast performance using vintage data and summarizing with the more robust geometric mean.}\label{fig:fcast-adjusted}
\end{figure}

\clearpage

\begin{figure}

{\centering \includegraphics[width=\textwidth]{fig/fcast-booted-1} 

}

\caption{Forecast performance when indicators are replaced with samples from their empirical distribution. Performance is largely similar to the AR model.}\label{fig:fcast-booted}
\end{figure}

\clearpage

\begin{figure}

{\centering \includegraphics[width=\textwidth]{fig/fcast-booted-adjusted-1} 

}

\caption{Forecast performance as measured with the geometric mean when indicators are replaced with samples from their empirical distribution. Performance is largely similar to the AR model.}\label{fig:fcast-booted-adjusted}
\end{figure}

\clearpage

\begin{figure}

{\centering \includegraphics[width=\textwidth]{fig/hot-booted-1} 

}

\caption{Hotspot prediction performance when indicators are replaced with samples from their empirical distribution. Performance is largely similar to the AR model.}\label{fig:hot-booted}
\end{figure}

\clearpage

\begin{figure}

{\centering \includegraphics[width=\textwidth]{fig/cor-wis-ratio-1} 

}

\caption{This is one of the correlation plots Alden made. It shows histograms of the Spearman correlation between the ratio of AR to AR WIS with the percent change in 7dav cases relative to 7 days earlier.}\label{fig:cor-wis-ratio}
\end{figure}

\clearpage

\begin{figure}

{\centering \includegraphics[width=\textwidth]{fig/cor-wis-ratio-m1-1} 

}

\caption{This is Alden's second set of histograms. Here we have the correlation of the absolute value of WIS ratio - 1 with the percent change in 7dav cases relative to 7 days earlier}\label{fig:cor-wis-ratio-m1}
\end{figure}

\clearpage

\begin{figure}

{\centering \includegraphics[width=\textwidth]{fig/upswing-histogram-logged-1} 

}

\caption{Not sure if we want this here. Similar to Figure 5 in the manuscript but taking logs. }\label{fig:upswing-histogram-logged}
\end{figure}

\clearpage

\begin{table}

\caption{\label{tab:upswing-corr-table}Correlation of the difference in WIS between the AR model with the difference in median predictions. In down periods, improvements in forecast risk are highly correlated with lower median predictions. The opposite is true in up periods. This suggests, as one might expect that improved performance of the indicator-assisted model is attributable to being closer to the truth then the AR model. This conclusion is stronger in down periods then in up periods.}
\centering
\begin{tabular}[t]{lrrrrr}
\toprule
udf & CHNG-CLI & CHNG-COVID & CTIS-CLIIC & DV-CLI & Google-AA\\
\midrule
down & 0.79 & 0.80 & 0.83 & 0.81 & 0.81\\
flat & 0.12 & 0.17 & 0.28 & 0.18 & 0.17\\
up & -0.57 & -0.56 & -0.53 & -0.53 & -0.47\\
\bottomrule
\end{tabular}
\end{table}

\clearpage

\begin{figure}

{\centering \includegraphics[width=\textwidth]{fig/hotspots-upswing-downswing-1} 

}

\caption{Classification and loglikelihood separated into periods of upswing, downswing, and flat cases. Like the analysis of the forecasting task in the main paper (see Figure 7), performance is better during down and flat periods.}\label{fig:hotspots-upswing-downswing}
\end{figure}

\clearpage

\begin{figure}

{\centering \includegraphics[width=\textwidth]{fig/fcast-alldates-1} 

}

\caption{Forecast performance over all periods. Performance largely improves for all forecasters with the inclusion of data in 2021.}\label{fig:fcast-alldates}
\end{figure}

\clearpage

\begin{figure}

{\centering \includegraphics[width=\textwidth]{fig/fcast-alldates-adjusted-1} 

}

\caption{Forcast performance over all periods aggregaged with the geometric mean. Again, the inclusion of data in 2021 leads to improved performance.}\label{fig:fcast-alldates-adjusted}
\end{figure}

\clearpage

\begin{figure}

{\centering \includegraphics[width=\textwidth]{fig/hot-alldates-1} 

}

\caption{Area under the curve for hotspot predictions including data in 2021. Performance degrades relative to the period in 2020. However, there are far fewer hotspots during this period as case rates declined in much of the country.}\label{fig:hot-alldates}
\end{figure}

\clearpage

\clearpage

\FloatBarrier

\movie{Type legend for the movie here.}

\dataset{dataset_one.txt}{Type or paste legend here.}

\dataset{dataset_two.txt}{Type or paste legend here. Adding longer text to show what happens, to decide on alignment and/or indentations for multi-line or paragraph captions.}






\bibliography{../../common/covidcast.bib,pnas-materials/pnas-sample.bib}

\end{document}