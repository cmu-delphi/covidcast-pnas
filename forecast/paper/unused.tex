
%%%%%


We also investigate the effect of lengthening the time window to include 2021. 
\begin{itemize}
\item For forecasting: results get better, whether we look at
  non-adjusted (first plot) or adjusted scores (second plot).  Other
  than that, interpretation is qualitatively similar, except fb now
  seems to be worse than gs.  And \chngcli~is strong, even in the
  non-adjusted view.
  \item For hotspots: Alden will write something explaining why we
    don't do hotspot detection in 2021.
\end{itemize}

%%%%% 

We look at various approaches to
accounting for the fact that the \gs~indicator is only available
for XX HRRs.

\begin{itemize}
  \item For forecasting task: clearly gs is not missing at random, and when it's present, it tends to be predictive.  Hence high values have a low false positivity rate.  Most visible in the adjusted view below, where gs triumphs.  Also \chngcli~gets a lot worse.

\item For hotspots: All methods look better, suggesting that the “GS locations” are easier for hotspot prediction. E.g. at 7-days ahead, the AUCs computed based on all locations range from 0.61-0.66; restricted to GS locations, the AUCs range from about 0.65-0.69. (These are all eyeballed, should get actual numbers if we want to say something like this in paper.)
\item gs\_subset appears to be particularly helped by this subsetting
  (which makes sense since on those other locations it was 0-imputed)
  \item Daniel will also try gs\_impute as in forecasting
\end{itemize}

%%%%%


